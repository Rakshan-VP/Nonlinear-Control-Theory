\chapterimage{orange2.jpg} % Chapter heading image
\chapterspaceabove{6.75cm} % Whitespace from the top of the page to the chapter title on chapter pages
\chapterspacebelow{7.25cm} % Amount of vertical whitespace from the top margin to the start of the text on chapter pages

\chapter{Nonlinear Observers}\index{Nonlinear Observers}
\section{Overview}\index{Overview}
In previous chapters, it was assumed that the system state vector $x$ was directly measurable and available for control design. However, in most practical situations, not all state variables can be measured, making it necessary to reconstruct the state from available outputs. This process, known as \textbf{state estimation}, relies on certain \textbf{observability conditions} being satisfied, allowing the use of an \textbf{observer} to estimate the true state. While observer design for linear time-invariant systems is well established, extending these concepts to nonlinear systems remains a challenging task. This chapter introduces the fundamentals of observer design and presents key results applicable to nonlinear systems.