\chapterimage{orange2.jpg} % Chapter heading image
\chapterspaceabove{6.75cm} % Whitespace from the top of the page to the chapter title on chapter pages
\chapterspacebelow{7.25cm} % Amount of vertical whitespace from the top margin to the start of the text on chapter pages

\chapter{Feedback Systems}\index{Feedback Systems}

\section{Overview}\index{Overview}

Until now, we have focused on open-loop systems. In this chapter, our attention shifts to \textbf{closed-loop systems}, where feedback is used to stabilize and control the system.  

We will introduce basic \textbf{feedback stabilization} and the \textbf{backstepping method}. Consider a general nonlinear system:
\[
\dot{x} = f(x,u), \quad x \in \mathbb{R}^n, \; u \in \mathbb{R}^m,
\]
where \(u\) is the control input.  

In a \textbf{state feedback} approach, the control is chosen as a function of the state:
\[
u = \phi(x),
\]
so that the closed-loop system becomes
\[
\dot{x} = f(x, \phi(x)).
\]

The goal of feedback design is to choose \(\phi(x)\) such that the equilibrium \(x=0\) (or any desired state) is \textbf{stable}, preferably \textbf{asymptotically} or \textbf{exponentially stable}.

\section{Basic Feedback Stabilization}\index{Basic Feedback Stabilization}

We illustrate feedback stabilization using simple examples and explain the impact of nonlinear terms.  

\begin{example}[Exact Cancellation – Globally Asymptotically Stable]
Consider the system:
\[
\dot{x} = a x^2 + u.
\]

\begin{itemize}
    \item \textbf{Step 1: Choose Control Input:}  
    Suppose we can exactly cancel the nonlinear term by choosing
    \[
    u = - a x^2 - x.
    \]

    \item \textbf{Step 2: Closed-Loop Dynamics:}  
    Substituting the control gives
    \[
    \dot{x} = -x.
    \]

    \item \textbf{Step 3: Stability Analysis:}  
    \begin{itemize}
        \item Take a Lyapunov function \(V(x) = \frac{1}{2} x^2\).  
        \item Then \(\dot{V}(x) = x \dot{x} = - x^2 \le 0\).  
        \item Therefore, \(x=0\) is \textbf{globally asymptotically stable (GAS)}.
    \end{itemize}
\end{itemize}
\end{example}

\begin{remark}[\textbf{Practical Limitation}]
In practice, exact cancellation of the nonlinear term \(a x^2\) may not be possible. For instance, suppose the implemented control is
\[
u = - \bar{a} x^2 - x,
\]
with \(\bar{a} \neq a\). Then the closed-loop system becomes
\[
\dot{x} = (a - \bar{a}) x^2 - x.
\]

\begin{itemize}
    \item The term \((a-\bar{a}) x^2\) may cause the system to be only \textbf{locally asymptotically stable (LAS)}.  
    \item This illustrates that exact cancellation is not always required or achievable in real systems.
\end{itemize}
\end{remark}

\begin{example}[Nonlinear Terms Can Be Beneficial – Detailed Analysis]
Consider the system:
\[
\dot{x} = a x^2 - x^3 + u,
\]
where \(x \in \mathbb{R}\) and \(u\) is the control input.  

\begin{itemize}
    \item \textbf{Step 1: Choose a simple control input:}  
    Let us select
    \[
    u = - a x^2 - x.
    \]
    This eliminates the \(a x^2\) term and introduces linear damping. The closed-loop system becomes:
    \[
    \dot{x} = - x - x^3.
    \]

    \item \textbf{Step 2: Observe the remaining nonlinear term:}  
    The cubic term \(-x^3\) remains. Unlike the \(a x^2\) term, it \textbf{assists in stabilization} because it is negative for positive \(x\) and positive for negative \(x\), i.e.,
    \[
    -x^3 \cdot x = -x^4 \le 0.
    \]

    \item \textbf{Step 3: Construct a Lyapunov function:}  
    A natural choice is the quadratic function
    \[
    V(x) = \frac{1}{2} x^2, \quad V(x) \ge 0, \quad V(0)=0.
    \]

    \item \textbf{Step 4: Compute the derivative along the trajectories:}  
    \[
    \dot{V}(x) = \frac{dV}{dx} \dot{x} = x \cdot (-x - x^3) = -x^2 - x^4.
    \]

    \item \textbf{Step 5: Analyze the derivative:}  
    \begin{itemize}
        \item For all \(x \neq 0\), \(\dot{V}(x) = -x^2 - x^4 < 0\).  
        \item This shows that the energy \(V(x)\) decreases along system trajectories.  
        \item Therefore, the equilibrium \(x=0\) is \textbf{globally asymptotically stable (GAS)}.
    \end{itemize}

    \item \textbf{Step 6: Insights from the nonlinear term:}  
    \begin{itemize}
        \item The term \(-x^3\) naturally adds damping for large \(x\), making the system return faster to the origin.  
        \item This demonstrates that nonlinear terms are not always detrimental—they can enhance global stability.  
    \end{itemize}
\end{itemize}
\end{example}

\begin{remark}[\textbf{Alternative Lyapunov-Based Design}]
When exact cancellation of nonlinear terms is not possible, an alternative approach is to:

\begin{itemize}
    \item Construct a Lyapunov function \(V_1(x)\), e.g., \(V_1(x) = \frac{1}{2} x^2\).  
    \item Design the control \(u(x)\) such that
    \[
    \dot{V}_1(x) = \frac{d V_1}{dx} \dot{x} \le - W(x), \quad W(x) > 0 \;\; \forall x \neq 0.
    \]  
    \item This ensures asymptotic stability without needing exact cancellation of all nonlinear terms.  
    \item Nonlinear terms that naturally help damping can be retained, simplifying the controller.
\end{itemize}
\end{remark}

\section{Integrator Backstepping}\index{Integrator Backstepping}
\begin{figure}[h!]
    \centering
    
    % ====== (a) ======
    \begin{tikzpicture}[auto, node distance=1.5cm]
    \tikzset{
      block/.style={draw, rectangle, minimum height=1cm, minimum width=1cm},
      sum/.style={draw, circle, inner sep=1pt, minimum size=6mm},
      line/.style={->, thick}
    }

    \node (u) {$u$};
    \node[block, right=of u] (int1) {$\int$};
    \node[block, right=of int1] (g) {$g(x)$};
    \node[sum, right=of g] (sum1) {$+$};
    \node[block, right=of sum1] (int2) {$\int$};
    \node[right=of int2] (x) {$x$};
    \node[block, below=1.5cm of int2] (f) {$f(\cdot)$};

    \draw[line] (u) -- (int1);
    \draw[line] (int1) -- node[above] {$\xi$} (g);
    \draw[line] (g) -- (sum1);
    \draw[line] (sum1) -- (int2);
    \draw[line] (int2) -- (x);
    \draw[line] (x) |- (f);
    \draw[line] (f) -| (sum1);

    \node at ($(u)!0.5!(x)+(0,1)$) {(a)};
    \end{tikzpicture}
    
    \vspace{1cm}
    
    % ====== (b) ======
    \begin{tikzpicture}[auto, node distance=1.5cm]
    \tikzset{
      block/.style={draw, rectangle, minimum height=1cm, minimum width=1cm},
      sum/.style={draw, circle, inner sep=1pt, minimum size=6mm},
      line/.style={->, thick}
    }

    \node (u) {$u$};
    \node[block, right=of u] (int1) {$\int$};
    \node[sum, right=of int1] (sum0) {$+$};
    \node[block, right=of sum0] (g) {$g(x)$};
    \node[sum, right=of g] (sum1) {$+$};
    \node[block, right=of sum1] (int2) {$\int$};
    \node[right=of int2] (x) {$x$};
    \node[block, below=1.5cm of int2] (f) {$f(\cdot)+g(\cdot)\phi(\cdot)$};

    \draw[line] (u) -- (int1);
    \draw[line] (int1) -- node[above] {$\xi$} (sum0);
    \draw[line] (sum0) -- (g);
    \draw[line] (g) -- (sum1);
    \draw[line] (sum1) -- (int2);
    \draw[line] (int2) -- (x);
    \draw[line] (x) |- (f);
    \draw[line] (f) -| (sum1);

    \draw[line] ($(sum0.south)+(0,-0.3)$) node[below] {$-\phi(x)$} -- (sum0.south);

    \node at ($(u)!0.5!(x)+(0,1)$) {(b)};
    \end{tikzpicture}
    
    \vspace{1cm}
    
    % ====== (c) ======
    \begin{tikzpicture}[auto, node distance=1.5cm]
    \tikzset{
      block/.style={draw, rectangle, minimum height=1cm, minimum width=1cm},
      sum/.style={draw, circle, inner sep=1pt, minimum size=6mm},
      line/.style={->, thick}
    }

    \node (u) {$u$};
    \node[sum, right=of u] (sum0) {$+$};
    \node[block, right=of sum0] (int1) {$\int$};
    \node[block, right=of int1] (g) {$g(x)$};
    \node[sum, right=of g] (sum1) {$+$};
    \node[block, right=of sum1] (int2) {$\int$};
    \node[right=of int2] (x) {$x$};
    \node[block, below=1.5cm of int2] (f) {$f(\cdot)+g(\cdot)\phi(\cdot)$};

    \draw[line] (u) -- (sum0);
    \draw[line] (sum0) -- (int1);
    \draw[line] (int1) -- node[above] {$z$} (g);
    \draw[line] (g) -- (sum1);
    \draw[line] (sum1) -- (int2);
    \draw[line] (int2) -- (x);
    \draw[line] (x) |- (f);
    \draw[line] (f) -| (sum1);

    \draw[line] ($(sum0.south)+(0,-0.3)$) node[below] {$-\phi(x)$} -- (sum0.south);

    \node at ($(u)!0.5!(x)+(0,1)$) {(c)};
    \end{tikzpicture}
    
    \vspace{1cm}
    
    % ====== (d) ======
    \begin{tikzpicture}[auto, node distance=1.5cm]
    \tikzset{
      block/.style={draw, rectangle, minimum height=1cm, minimum width=1cm},
      sum/.style={draw, circle, inner sep=1pt, minimum size=6mm},
      line/.style={->, thick}
    }

    \node (v) {$v=\dot{z}$};
    \node[block, right=of v] (int1) {$\int$};
    \node[block, right=of int1] (g) {$g(x)$};
    \node[sum, right=of g] (sum1) {$+$};
    \node[block, right=of sum1] (int2) {$\int$};
    \node[right=of int2] (x) {$x$};
    \node[block, below=1.5cm of int2] (f) {$f(\cdot)+g(\cdot)\phi(\cdot)$};

    \draw[line] (v) -- (int1);
    \draw[line] (int1) -- node[above] {$z$} (g);
    \draw[line] (g) -- (sum1);
    \draw[line] (sum1) -- (int2);
    \draw[line] (int2) -- (x);
    \draw[line] (x) |- (f);
    \draw[line] (f) -| (sum1);

    \node at ($(v)!0.5!(x)+(0,1)$) {(d)};
    \end{tikzpicture}
    
    \caption{Block diagram representations: (a) Basic structure, (b) With $\phi(x)$ cancellation, (c) Alternate structure, (d) Reformulated with $v=\dot{z}$.}
    \label{fig:block_diagrams}
\end{figure}

\begin{itemize}
    \item \textbf{System in cascade form:}
    \[
        \dot{x} = f(x) + g(x)\, \xi  
    \]
    \[
        \dot{\xi} = u
    \]

    \item \textbf{Virtual control:} Assume $\xi = \phi(x)$ stabilizes the $x$-subsystem with Lyapunov function $V_1(x)$:
    \[
        \dot{V}_1(x) \le -V_a(x) \le 0
    \]

    \item \textbf{Expose the error:} Add and subtract $g(x)\phi(x)$:
    \[
        \dot{x} = f(x) + g(x)\phi(x) + g(x)(\xi - \phi(x))
    \]

    \item \textbf{Define tracking error:}
    \[
        z := \xi - \phi(x), \quad \dot{z} = u - \dot{\phi}(x), \quad 
        \dot{\phi}(x) = \frac{\partial \phi}{\partial x}[f(x) + g(x)\xi]
    \]

    \item \textbf{Introduce new control:} Set $v := \dot{z}$. Then
    \[
        \dot{x} = f(x) + g(x)\phi(x) + g(x) z, \quad \dot{z} = v
    \]

    \item \textbf{Composite Lyapunov candidate:}
    \[
        V(x,z) = V_1(x) + \frac{1}{2} z^2
    \]

    \item \textbf{Design $v$ to stabilize:}
    \[
        v = - \left( \frac{\partial V_1}{\partial x} g(x) + k z \right), \quad k>0
    \]
    \[
        \dot{V} \le - V_a(x) - k z^2 \le 0
    \]

    \item \textbf{Recover actual control:}
    \[
        u = v + \dot{\phi}(x)
          = - \left( \frac{\partial V_1}{\partial x} g(x) + k z \right) + \frac{\partial \phi}{\partial x}[f(x) + g(x)\xi]
    \]

    \[
    \boxed{
    u = - \frac{\partial V_1}{\partial x} g(x) - k (\xi - \phi(x)) + \frac{\partial \phi}{\partial x} \big[ f(x) + g(x) \xi \big]
    }
    \]
    

    \item \textbf{Conclusion:} With $z \to 0$ and the $x$-subsystem stabilized, $(x,\xi)$ is asymptotically stable. If $V_1$ is radially unbounded, stability can be global.
\end{itemize}





\section{Backstepping: More General Cases}\index{Backstepping: More General Cases}

Backstepping is a recursive design methodology used to stabilize systems in \textit{strict-feedback form} or chains of integrators. It constructs control laws step by step while designing Lyapunov functions to guarantee stability.

\subsection{Chain of Integrators}\index{Backstepping: More General Cases!Chain of integrators}

Consider a system in the form of a chain of integrators:
\begin{align}
    \dot{x} &= f(x) + g(x)\xi_1, \\
    \dot{\xi}_1 &= \xi_2, \\
    \dot{\xi}_2 &= \xi_3, \\
    &\;\;\vdots \notag\\
    \dot{\xi}_{r-1} &= \xi_r, \\
    \dot{\xi}_r &= u,
\end{align}
where $x \in \mathbb{R}^n$, $\xi_i \in \mathbb{R}$, and $u \in \mathbb{R}$.  

Assume the \textit{reduced system}:
\[
\dot{x} = f(x) + g(x)\phi(x)
\]
is stabilized by the virtual control $\xi_1 = \phi(x)$, with a Lyapunov function $V_1(x)$ such that
\[
\dot{V}_1(x) = L_f V_1(x) + L_g V_1(x) \phi(x) \leq -W_1(x), \quad W_1(x) > 0.
\]

\begin{itemize}
    
\item\textbf{Step 1: Define first error variable}

\begin{itemize}
    \item Introduce the error:
    \[
        z_1 = \xi_1 - \phi(x).
    \]
    \item Extend the Lyapunov function:
    \[
        V_2(x, z_1) = V_1(x) + \frac{1}{2} z_1^2.
    \]
    \item Its time derivative is:
    \begin{align*}
        \dot{V}_2 &= \dot{V}_1(x) + z_1 \dot{z}_1 \\
        &= L_f V_1(x) + L_g V_1(x)\xi_1 + z_1(\xi_2 - \dot{\phi}(x)) \\
        &= L_f V_1(x) + L_g V_1(x)\phi(x) + L_g V_1(x) z_1 + z_1(\xi_2 - \dot{\phi}(x)) \\
        &\leq -W_1(x) + z_1\big(L_g V_1(x) + \xi_2 - \dot{\phi}(x)\big).
    \end{align*}
\end{itemize}

\item\textbf{Step 2: Introduce next error variable}

\begin{itemize}
    \item Define:
    \[
        z_2 = \xi_2 - \alpha_1(x, \xi_1),
    \]
    where $\alpha_1(x,\xi_1)$ is designed to cancel the troublesome term:
    \[
        \alpha_1(x, \xi_1) = -L_g V_1(x) + \dot{\phi}(x) - k_1 z_1, \quad k_1 > 0.
    \]
    \item Extend the Lyapunov function:
    \[
        V_3 = V_2 + \frac{1}{2} z_2^2.
    \]
    \item Its derivative:
    \[
        \dot{V}_3 = -W_1(x) - k_1 z_1^2 + z_2(\xi_3 - \dot{\alpha}_1).
    \]
\end{itemize}

\item\textbf{Step 3: Recursive design}

Proceed recursively for $k=3,4,\dots,r$:
\begin{align}
    z_k &= \xi_k - \alpha_{k-1}(x,\xi_1,\dots,\xi_{k-1}), \\
    V_{k+1} &= V_k + \frac{1}{2} z_k^2, \\
    \dot{V}_{k+1} &\leq - W_k(x, z_1, \dots, z_{k-1}) - k_{k-1} z_{k-1}^2 + z_k(\xi_{k+1} - \dot{\alpha}_{k-1}).
\end{align}

\item\textbf{Final Step: Control input}

\begin{itemize}
    \item At $k=r$, define the actual control input:
    \[
        u = \alpha_r(x, \xi_1, \dots, \xi_r),
    \]
    with
    \[
        \alpha_r = \dot{\alpha}_{r-1} - k_r z_r.
    \]
    \item The resulting derivative satisfies
    \[
        \dot{V}_{r+1} \leq - W(x, z_1, \dots, z_r),
    \]
    which guarantees asymptotic stability of the full system.
\end{itemize}
\end{itemize}

\subsection{Strict-Feedback Systems}\index{Backstepping: More General Cases!Strict Feedback Systems}

For systems in strict-feedback form:
\begin{align}
    \dot{\xi}_1 &= f_1(\xi_1) + g_1(\xi_1) \xi_2, \\
    \dot{\xi}_2 &= f_2(\xi_1, \xi_2) + g_2(\xi_1, \xi_2) \xi_3, \\
    &\;\vdots \notag\\
    \dot{\xi}_r &= f_r(\xi_1, \dots, \xi_r) + g_r(\xi_1, \dots, \xi_r) u,
\end{align}
the backstepping procedure is applied recursively:
\begin{itemize}
    \item Define virtual control laws $\alpha_i$ and errors $z_i$ step by step.
    \item Construct Lyapunov functions $V_{i+1} = V_i + \frac{1}{2} z_i^2$ at each step.
    \item At the last step, define $u = \alpha_r$ to stabilize the full system.
\end{itemize}

\begin{example}[Magnetic Suspension System with Constant Current]



Consider a simplified magnetic levitation system:
\begin{align}
    \dot{x}_1 &= x_2, \\
    \dot{x}_2 &= g - \frac{k}{m}\frac{I^2}{(x_1 + c)^2},
\end{align}
where $x_1$ is the position of the levitated object, $x_2$ its velocity, $I$ the constant coil current, and $g, k, m, c$ are constants.  

\begin{center}
\begin{tikzpicture}[auto, node distance=3cm, thick]

    % Blocks
    \node[draw, rectangle, minimum width=8.5cm, minimum height=1cm] (C) {Backstepping Controller: $\alpha_1(x_1)=-k_1 z_1$};
    \node[draw, rectangle, minimum width=8cm, minimum height=1cm, below=of C] (P) {Magnetic Suspension: $\dot{x}_1=x_2$, $\dot{x}_2=g-\frac{k}{m}\frac{I^2}{(x_1+c)^2}$};

    % Input and output
    \node[left=of C] (ref) {Error $z_{1}$};
    \node[left=of P] (out) {Position $x_1$};

    % Arrows
    \draw[->] (ref) -- (C) node[midway, above] {$z_1 = x_1 - x_{1d}$};
    \draw[->] (C) -- (P) node[midway, right] {Control $\alpha_1$};
    \draw[->] (P) -- (out);
    \draw[->, dashed] (out.north) --  (ref.south) node[pos=0.5, right] {Feedback};

\end{tikzpicture}
\end{center}

\begin{itemize}
    \item \textbf{Step 1: Define position error.} Let
    \[
        z_1 = x_1 - x_{1d},
    \]
    where $x_{1d}$ is the desired position. Lyapunov function:
    \[
        V_1 = \frac{1}{2} z_1^2.
    \]

    \item \textbf{Step 2: Define velocity error.} Let
    \[
        z_2 = x_2 - \alpha_1(x_1),
    \]
    where $\alpha_1(x_1)$ is a virtual control law chosen to stabilize $z_1$, for example:
    \[
        \alpha_1(x_1) = -k_1 z_1, \quad k_1>0.
    \]
    
    Extend the Lyapunov function:
    \[
        V_2 = V_1 + \frac{1}{2} z_2^2.
    \]

    \item \textbf{Step 3: Compute derivative and verify stability.}  
    \begin{align*}
        \dot{z}_1 &= z_2 + \alpha_1(x_1), \\
        \dot{z}_2 &= g - \frac{k}{m}\frac{I^2}{(x_1+c)^2} - \dot{\alpha}_1(x_1).
    \end{align*}
    
    Choosing $\alpha_1(x_1)$ properly ensures
    \[
        \dot{V}_2 = z_1 \dot{z}_1 + z_2 \dot{z}_2 \leq - k_1 z_1^2 - k_2 z_2^2 \leq 0.
    \]

\end{itemize}

Thus, the simplified 2-state system is asymptotically stabilized using backstepping.

\end{example}





