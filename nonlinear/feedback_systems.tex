\chapterimage{orange2.jpg} % Chapter heading image
\chapterspaceabove{6.75cm} % Whitespace from the top of the page to the chapter title on chapter pages
\chapterspacebelow{7.25cm} % Amount of vertical whitespace from the top margin to the start of the text on chapter pages

\chapter{Feedback Systems}\index{Feedback Systems}

\section{Overview}\index{Overview}

Until now, we have focused on open-loop systems. In this chapter, our attention shifts to \textbf{closed-loop systems}, where feedback is used to stabilize and control the system.  

We will introduce basic \textbf{feedback stabilization} and the \textbf{backstepping method}. Consider a general nonlinear system:
\[
\dot{x} = f(x,u), \quad x \in \mathbb{R}^n, \; u \in \mathbb{R}^m,
\]
where \(u\) is the control input.  

In a \textbf{state feedback} approach, the control is chosen as a function of the state:
\[
u = \phi(x),
\]
so that the closed-loop system becomes
\[
\dot{x} = f(x, \phi(x)).
\]

The goal of feedback design is to choose \(\phi(x)\) such that the equilibrium \(x=0\) (or any desired state) is \textbf{stable}, preferably \textbf{asymptotically} or \textbf{exponentially stable}.

\section{Basic Feedback Stabilization}\index{Basic Feedback Stabilization}

We illustrate feedback stabilization using simple examples and explain the impact of nonlinear terms.  

\begin{example}[Exact Cancellation – Globally Asymptotically Stable]
Consider the system:
\[
\dot{x} = a x^2 + u.
\]

\begin{itemize}
    \item \textbf{Step 1: Choose Control Input:}  
    Suppose we can exactly cancel the nonlinear term by choosing
    \[
    u = - a x^2 - x.
    \]

    \item \textbf{Step 2: Closed-Loop Dynamics:}  
    Substituting the control gives
    \[
    \dot{x} = -x.
    \]

    \item \textbf{Step 3: Stability Analysis:}  
    \begin{itemize}
        \item Take a Lyapunov function \(V(x) = \frac{1}{2} x^2\).  
        \item Then \(\dot{V}(x) = x \dot{x} = - x^2 \le 0\).  
        \item Therefore, \(x=0\) is \textbf{globally asymptotically stable (GAS)}.
    \end{itemize}
\end{itemize}
\end{example}

\begin{remark}[\textbf{Practical Limitation}]
In practice, exact cancellation of the nonlinear term \(a x^2\) may not be possible. For instance, suppose the implemented control is
\[
u = - \bar{a} x^2 - x,
\]
with \(\bar{a} \neq a\). Then the closed-loop system becomes
\[
\dot{x} = (a - \bar{a}) x^2 - x.
\]

\begin{itemize}
    \item The term \((a-\bar{a}) x^2\) may cause the system to be only \textbf{locally asymptotically stable (LAS)}.  
    \item This illustrates that exact cancellation is not always required or achievable in real systems.
\end{itemize}
\end{remark}

\begin{example}[Nonlinear Terms Can Be Beneficial – Detailed Analysis]
Consider the system:
\[
\dot{x} = a x^2 - x^3 + u,
\]
where \(x \in \mathbb{R}\) and \(u\) is the control input.  

\begin{itemize}
    \item \textbf{Step 1: Choose a simple control input:}  
    Let us select
    \[
    u = - a x^2 - x.
    \]
    This eliminates the \(a x^2\) term and introduces linear damping. The closed-loop system becomes:
    \[
    \dot{x} = - x - x^3.
    \]

    \item \textbf{Step 2: Observe the remaining nonlinear term:}  
    The cubic term \(-x^3\) remains. Unlike the \(a x^2\) term, it \textbf{assists in stabilization} because it is negative for positive \(x\) and positive for negative \(x\), i.e.,
    \[
    -x^3 \cdot x = -x^4 \le 0.
    \]

    \item \textbf{Step 3: Construct a Lyapunov function:}  
    A natural choice is the quadratic function
    \[
    V(x) = \frac{1}{2} x^2, \quad V(x) \ge 0, \quad V(0)=0.
    \]

    \item \textbf{Step 4: Compute the derivative along the trajectories:}  
    \[
    \dot{V}(x) = \frac{dV}{dx} \dot{x} = x \cdot (-x - x^3) = -x^2 - x^4.
    \]

    \item \textbf{Step 5: Analyze the derivative:}  
    \begin{itemize}
        \item For all \(x \neq 0\), \(\dot{V}(x) = -x^2 - x^4 < 0\).  
        \item This shows that the energy \(V(x)\) decreases along system trajectories.  
        \item Therefore, the equilibrium \(x=0\) is \textbf{globally asymptotically stable (GAS)}.
    \end{itemize}

    \item \textbf{Step 6: Insights from the nonlinear term:}  
    \begin{itemize}
        \item The term \(-x^3\) naturally adds damping for large \(x\), making the system return faster to the origin.  
        \item This demonstrates that nonlinear terms are not always detrimental—they can enhance global stability.  
    \end{itemize}
\end{itemize}
\end{example}

\begin{remark}[\textbf{Alternative Lyapunov-Based Design}]
When exact cancellation of nonlinear terms is not possible, an alternative approach is to:

\begin{itemize}
    \item Construct a Lyapunov function \(V_1(x)\), e.g., \(V_1(x) = \frac{1}{2} x^2\).  
    \item Design the control \(u(x)\) such that
    \[
    \dot{V}_1(x) = \frac{d V_1}{dx} \dot{x} \le - W(x), \quad W(x) > 0 \;\; \forall x \neq 0.
    \]  
    \item This ensures asymptotic stability without needing exact cancellation of all nonlinear terms.  
    \item Nonlinear terms that naturally help damping can be retained, simplifying the controller.
\end{itemize}
\end{remark}

\section{Integrator Backstepping}\index{Integrator Backstepping}
\section{Backstepping: More General Cases}\index{Backstepping: More General Cases}
\subsection{Chain of integrators}\index{Backstepping: More General Cases!Chain of integrators}
\subsection{Strict Feedback Systems}\index{Backstepping: More General Cases!Strict Feedback Systems}