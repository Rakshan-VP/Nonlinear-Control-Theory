\chapterimage{orange2.jpg} % Chapter heading image
\chapterspaceabove{6.75cm} % Whitespace from the top of the page to the chapter title on chapter pages
\chapterspacebelow{7.25cm} % Amount of vertical whitespace from the top margin to the start of the text on chapter pages

\chapter{Feedback Linearisation}\index{Feedback Linearisation}
\section{Overview}\index{Overview}
This chapter introduces the concept of \textbf{feedback linearization}, a nonlinear control technique used to transform a nonlinear system into an equivalent \textbf{linear time-invariant (LTI)} form through an appropriate control law and change of variables. Once linearized, standard linear control methods can be applied to achieve desired system performance. The method, developed extensively during the 1970s and 1980s, has proven effective but also faces limitations such as sensitivity to modeling errors and structural constraints. Despite these challenges, feedback linearization remains a \textbf{fundamental tool in nonlinear control theory}, providing valuable insight into the control of complex dynamic systems.